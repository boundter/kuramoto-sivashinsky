\section{Results}
\subsection{$M$ - Modes to consider}
\begin{figure}
  \centering
  \includegraphics[width=0.66\textwidth]{modes}
  \caption{Evolution of the magnitude of modes for $L=40$.}\label{fig:modes}
\end{figure}
First we need to know, how many modes we should consider in the simulation. The evolution of the modes for $L=40$ is plotted in figure \ref{fig:modes}. According to the linear approximation there will be $N=6$ evolving modes. As we can see in the figure, there are in reality about 20 modes. If we choose $M = 4N$ we will be on the safe side. Even if there are more modes, they will decrease drastically over time. This value will be used for all later simulations.

\subsection{Solutions for different $L$}
\begin{figure}
  \begin{subfigure}{0.5\textwidth}
    \includegraphics[width=\textwidth]{{{velocity_L6.29}}}
    \caption{$L=6.29$}\label{fig:629}
    \includegraphics[width=\textwidth]{{{velocity_L15.00}}}
    \caption{$L=15$}\label{fig:15}
  \end{subfigure}%
  \begin{subfigure}{0.5\textwidth}
    \includegraphics[width=\textwidth]{{{velocity_L30.00}}}
    \caption{$L=30$}\label{fig:30}
    \includegraphics[width=\textwidth]{{{velocity_L100.00}}}
    \caption{$L=100$}\label{fig:100}
  \end{subfigure}
  \caption{Evolution of the velocity field for different $L$.}\label{fig:L}
\end{figure}
In the figure \ref{fig:L} we can see the evolution of the velocity for different $L$. In \ref{fig:15} we can see clear patterns, the system reaches the stationary state quite fast. In figure \ref{fig:629} the evolution is not finished, but it will also reach the stationary state after some time.

But if we look at \ref{fig:30} and \ref{fig:100}, we will see another regime. The field is no longer stationary, but turbulent. For $L=30$ in \ref{fig:30} there are strong fluctuations and it is no longer periodic in time. In \ref{fig:100} there is also no periodicity. The system is in full turbulence.

\subsubsection{Dependence on $M$}
\begin{figure}
  \begin{subfigure}{0.5\textwidth}
    \includegraphics[width=\textwidth]{{{velocity_N16}}}
    \caption{$M=16$}\label{fig:16}
    \includegraphics[width=\textwidth]{{{velocity_N12}}}
    \caption{$M=12$}\label{fig:12}
  \end{subfigure}%
  \begin{subfigure}{0.5\textwidth}
    \includegraphics[width=\textwidth]{{{velocity_N8}}}
    \caption{$M=8$}\label{fig:8}
    \includegraphics[width=\textwidth]{{{velocity_N7}}}
    \caption{$M=7$}\label{fig:7}
  \end{subfigure}
  \caption{Evolution of the velocity field for $L=30$, but different a different amount of modes $M$.}\label{fig:M}
\end{figure}
The solutions we have found depend on the number of modes $M$. An increase in $M$ would not change much, since the higher modes die off quickly, but a decrease will have measurable consequences. An overview for $L=30$ is given in figure \ref{fig:M}. For $L=30$ we have a linear approximation of $N = 4$.

Between figure \ref{fig:16} and \ref{fig:12} we see a loss of the fine detail. The structure retains it turbulent state, but for the bigger $M$ the details are finer. Between \ref{fig:12} and \ref{fig:8} there is only a small change. And between \ref{fig:8} and \ref{fig:7} there is only one mode less and $M = 7 > N = 4$, yet we find that there are no turbulences.

\subsection{Particle advection}
\begin{figure}
  \centering
  \includegraphics[width=0.66\textwidth]{particles}
  \caption{Advection of particles in the velocity field. The particles are initially spaced two units apart from each other.}\label{fig:part}
\end{figure}
Now we place particles on the fluid and look at their advection for $L=60$, so we are in the turbulent regime. The particles are uniformly placed with a distance of two units. This is plotted in figure \ref{fig:part}. The evolution of the position $X$ is according to the equation
\begin{equation}
    \frac{d X}{dt} = u(x, t).
\end{equation}
The velocity field evolves using the ETD2RK algorithm, whereas the advection evolves using the Euler algorithm.

As we can see all the particles start at different positions, but after some time they all meet up. After about 100 time units they all have the same trajectory.

\subsection{Conclusion}
As we have seen, the Kuramoto-Sivashinsky equation describes two regimes, a stationary one and a turbulent one. The two regimes have been found numerically, where the turbulent regime starts somewhere above, but close by $L = 20$.

The accuracy of the simulation depends heavily on the number of Fourier modes, which are used for the evolution. Thanks to the Galerkin approximation there is a finite number of modes to consider, to describe the system accurately, but too strong a reduction lead to a loss of the turbulent regime.
