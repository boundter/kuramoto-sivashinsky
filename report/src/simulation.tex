\section{Simulation}
In general there are three parameters in simulating equation (\ref{eq:galerkin}). The number of modes to consider $M$, the spatial periodicity $L$ and the time step $h$. $M$ and $L$ were constantly varied and especially $M$ will be looked at later. The time step $h$ has been chosen to $0.01$.

The initial conditions were a random distribution of Fourier modes up to $M$ and the integral over $u$ has been chosen to zero, $C_0(t) = 0$, to reduce the superposition of velocity and turbulence. The DFT (Discreet Fourier transform) was used to get the velocity field from the modes according to (\ref{eq:inverse}).

The ODE (\ref{eq:galerkin}) is stiff, so we cannot use the Euler or Runge-Kutta algorithm. Instead we have to use the ETD2RK algorithm~\cite{cox2002exponential}, which reduces the equation to Runge-Kutta and is of order $O(h^2)$. The system describes an equation of form
\begin{equation}
  \frac{d C}{dt} = aC + F(C,t),
\end{equation}
where $C$ and $F$ are vectors and $a$ is a matrix. If $a$ is a diagonal matrix it can be treated as a scalar for the next calculations. The ETD2RK algorithm then reads
\begin{align}
  \tilde{C}_n & = C_{m,n} e^{ah} + F_{m,n} \frac{e^{ah} - 1}{a} \\
  C_{m + 1,n} & = \tilde{C_n} + \left(F_n(\tilde{C}_n, t_m + h) - F_n \right) \frac{e^{ah} - 1 - ah}{a^2 h},
\end{align}
where $n$ is the index of the vector and $m$ the index of time.
